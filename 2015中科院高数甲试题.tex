\documentclass[UTF8]{ctexart}
\usepackage{amsmath}
\usepackage{amsfonts}
\usepackage{amssymb}
\usepackage[a4paper,hmargin=1.25in,vmargin=1in]{geometry}
\date{}
\author{}
\linespread{2.0}
\begin{document}
\title{2015年中科院高等数学(甲)试题}
\maketitle
\thispagestyle{empty}
声明:这套高等数学(甲)试卷是由QQ1844120070的前辈在考场上记录下来的,原始照片很容易在网上搜到。现本人将其整理后发布出来,但因本人时间精力有限,故希望因本文档而受益的同学能做出解答,并将答案与网上一直流传的文件衔接起来,将分享精神传承下去。关于该文档的一切问题可以和上面的前辈联系,也可以向本人发邮件flaw.in.theory@gmail.com索要\LaTeX源码。


\noindent一.选择题\\
(1).对于函数$ f(x)=\dfrac{\sin{x^2}}{x} $.结论不正确的是:\\
\begin{tabular}{ll}
	A.在$ (0,\infty) $内有界.&
	B.在$ (0,\infty) $内$ f(x) $没有最大值和最小值.\\
	C.在$ (0.\infty) $内处处可导.&
	D.当$ x\rightarrow\infty $和$ x\rightarrow0^+ $时,$ f(x) $极限存在.
\end{tabular}


\noindent(2).$ \lim\limits_{x\rightarrow0}(\frac{\sin{x}}{x})^\frac{1}{x^2}= $(	)\\
\begin{tabular}{llll}
	A.$ 1 $& B.$ 0 $& C.$ \infty $& D.$ e^{-\frac{1}{6}} $
\end{tabular}


\noindent(3).微分方程$ y'=\dfrac{1}{y-x} $的通解为(	)\\
\begin{tabular}{ll}
	A.$ x=y+Ce^{-y}-1 $&
	B.$ y=x+Ce^{-x}-1 $\\
	C.$ x=\ln\left|x-y-1\right|+C $&
	D.$ y=\ln\left|y-x-1\right|+C$
\end{tabular}


\noindent(4).已知$ m,n $为正整数,且$ m>n $.如果:$ S=\int_{0}^{\frac{\pi}{4}}\sin^m{x}\cos^n{x}\mathrm{d}x,T=\int_{0}^{\frac{\pi}{4}}\sin^n{x}\cos^m{x}\mathrm{d}x $则下面结论正确的一个是(	)\\
\begin{tabular}{llll}
	A.$ S>T $& B.$ S=T $& C.$ S<T $& D.$ S,T $大小关系不确定
\end{tabular}


\noindent(5)设对任意的$ x\in\mathbb{R} $,总有$ m\le f(x)<g(x)<h(x)\le M $.且$ g(x) $为连续函数,若$ \lim\limits_{x\rightarrow\infty}[M-f(x)][h(x)-m]=0. $则对于$ \lim\limits_{x\rightarrow\infty}g(x) $,下面结论正确的一个是(	)\\
\begin{tabular}{ll}
	A.一定存在,且等于$ \dfrac{M+m}{2} $&
	B.一定存在,且只能等于$ M $或$ m $\\
	C.一定不存在&
	D.一定存在,且可以取到$ [m,M] $上的任意值
\end{tabular}


\noindent(6)设$ f(x)=x^2\sin{x}+\cos{x}+\frac{\pi}{2}x $,在其定义域内零点的个数是(	)\\
\begin{tabular}{llll}
	A.$ 2 $& B.$ 3 $& C.$ 4 $& D.多于4
\end{tabular}


\noindent(7)设$ \boldsymbol{a},\boldsymbol{b} $为非零实向量,且$ 2\boldsymbol{a}+\boldsymbol{b} $与$ \boldsymbol{a}-\boldsymbol{b} $垂直,$ \boldsymbol{a}+2\boldsymbol{b} $与$ \boldsymbol{a}+\boldsymbol{b} $垂直,则(	)\\
\begin{tabular}{llll}
	A.$ \left|\boldsymbol{b}\right|^2=7\left|\boldsymbol{a}\right|^2 $&
	B.$ \left|\boldsymbol{a}\right|^2=7\left|\boldsymbol{b}\right|^2 $&
	C.$ \left|\boldsymbol{b}\right|^2=5\left|\boldsymbol{a}\right|^2 $&
	D.$ \left|\boldsymbol{a}\right|^2=5\left|\boldsymbol{b}\right|^2 $
\end{tabular}

\thispagestyle{empty}


\noindent(8)设平面$ D $由$ x+y= \dfrac{1}{2},x+y=1 $及两条坐标轴围成.$ I_1=\underset{D}\iint\ln(x+y)^3\mathrm{d}x\mathrm{d}y,I_2=\underset{D}\iint(x+y)^3\mathrm{d}x\mathrm{d}y,I_3=\underset{D}\iint\sin(x+y)^3\mathrm{d}x\mathrm{d}y. $则下面结论正确的一个是(	)\\
\begin{tabular}{llll}
	A.$ I_1<I_2<I_3 $&
	B.$ I_3<I_1<I_2 $&
	C.$ I_1<I_3<I_2 $&
	D.$ I_3<I_2<I_1 $
\end{tabular}


\noindent(9)设幂级数$ \sum\limits_{n=1}^ \infty a_nx^n $与$ \sum\limits_{n=1}^{\infty}b_nx^n $的收敛半径分别为$ \dfrac{\sqrt{2}}{3} $与$ \dfrac{1}{3} $.则幂级数$ \sum\limits_{n=1}^{\infty}\frac{a_n^2}{b_n^2}x^n $的收敛半径为(	)\\
\begin{tabular}{llll}
	A.$ 2 $&
	B.$ \dfrac{\sqrt{2}}{3} $&
	C.$ \dfrac{1}{3} $&
	D.$ \dfrac{1}{2} $
\end{tabular}


\noindent(10)已知曲面$ x^2+y^2+2z^2=5 $,在其上点$ (x_0,y_0,z_0) $处的切平面与平面$ x+2y+z=0 $平分,则有(	)\\
\begin{tabular}{ll}
	A.$ x_0:y_0:z_0=4:2:1 $&
	B.$ x_0:y_0:z_0=2:4:1 $\\
	C.$ x_0:y_0:z_0=1:4:2 $&
	D.$ x_0:y_0:z_0=1:2:4 $
\end{tabular}


\noindent二.计算$\underset{n\rightarrow\infty}{\lim}\dfrac{1}{n}\left(\sqrt{1+\sin\dfrac{\pi}{n}}+\sqrt{1+\sin\dfrac{2\pi}{n}}+\text{……}+\sqrt{1+\sin\dfrac{n\pi}{n}}\right)$.


\noindent三.设$u=e^{x^{2}}\sin\dfrac{x}{y}$,计算$\dfrac{\partial^{2}u}{\partial x\partial y}$在点$\left(\pi,2\right)$处的值.


\noindent四.设$D$为第一象限内由$y=x,y=2x,xy=1,xy=2$所围成的区域,$f$为一元可微函数,且$f'=g$,记$L$为$D$的边界,证明:
\[\oint_{L}xf\left(\frac{y}{x}\right)\mathrm{d}x=-\int_{1}^{2}\frac{g\left(u\right)}{2u}\mathrm{d}u\].


\noindent五.已知$y_{1}=x,y_{2}=x^{2},y_{3}=e^x$为线性非齐次微分方程:$ y''+p(x)y'+q(x)y=f(x) $的三特解,求该方程满足初始条件$ y(0)=1,y'(0)=0 $的特解.


\noindent六.设$ f(x)=4x+\cos{\pi}{x}+\dfrac{1}{1+x^2}-x^2e^x+xe^x\int_{x}^{1}f(t)\mathrm{d}t $.求$ \int_{0}^{1}(1-x)e^xf(x)\mathrm{d}x $.


\noindent七.计算曲面积分$ I=\underset{\Sigma}\iint\dfrac{x\mathrm{d}y\mathrm{d}z+y\mathrm{d}z\mathrm{d}x+z\mathrm{d}x\mathrm{d}y}{(x^2+y^2+z^2)^\frac{3}{2}} $其中$ \Sigma $是曲面$ 2x^2+2y^2+z^2=4 $的外侧.


\noindent八.将函数$ f(x)=x-1,(0\le{x}\le{2}) $展开成周期为$ 4 $的余弦级数.


\noindent九.若$ g(x) $为单调增加的可微函数,且当$ x\ge{a} $时,$ \left|f'(x)\right|\le{g'(x)} $.证明:当$ x\ge{a} $时,$ \left|f(x)-f(a)\right|\le{g(x)-g(a)} $.


\noindent十.函数$ f(x) $在$ [0,2] $上二阶可导,且$ f'(0)=f'(2)=0 $.证明:在区间$ (0,2) $内至少存在一点$ \xi $,满足$ \left|f''(\xi)\right|\ge\left|f(2)-f(0)\right| $.


\noindent十一.设$ 0<a<1,x\ge0 $且$ y\ge0 $,证明: \\(1).$x^ay^{1-a}\le{ax+(1-a)y} $\\
(2).设 $x_1,x_2,\ldots x_n,y_1,y_2,\ldots y_n $均为实数,利用(1)的结果证明$ x_1y_1+x_2y_2+\ldots +x_ny_n\le (x_1^2+x_2^2+\ldots+x_n^2)^\frac{1}{2} (y_1^2+y_2^2+\ldots+y_n^2)^\frac{1}{2}$.
\thispagestyle{empty}
\end{document} 
