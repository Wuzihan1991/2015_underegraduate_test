\documentclass[UTF8]{ctexart}
\usepackage{geometry}
\geometry{verbose,tmargin=1.54cm,bmargin=1.54cm,lmargin=2cm,rmargin=2cm}
\usepackage{amsmath}
\usepackage{amsfonts}
\usepackage{amssymb}

\author{}
\date{}
\begin{document}

\title{2015年中科院考研试题量子力学}

\maketitle
%\thispagestyle{empty}

声明:这套量子力学试卷是由QQ1844120070的前辈在考场上记录下来的,原始照片很容易在网上搜到。现本人将其整理后发布出来,但因本人时间精力有限,故希望因本文档而受益的同学能做出解答,并将答案与网上一直流传的文件衔接起来,将分享精神传承下去。关于该文档的一切问题可以和上面的前辈联系,也可以向本人发邮件flaw.in.theory@gmail.com索要\LaTeX源码。


一.质量为$\mu$的粒子限制在长度为$L$的一维匣子$(0<x<L)$中自由运动,在$x=0$与$x=L$处其定态波函数满足条件$\Psi(0)=\Psi(L)$,$\Psi'(0)=\Psi'(L)$.
\begin{enumerate}
\item 求体系能级.
\item 将第一激发态归一化波函数表示为动量本征态的线性组合,并求动量平均值为$0$时组合系数满足的条件.
\end{enumerate}

二.粒子在球对称谐振子势阱$V\left(r\right)=\dfrac{1}{2}\mu\omega^{2}\left(x^{2}+y^{2}+z^{2}\right)$中运动,受到微扰作用$H'=\lambda\left(xyz+x^{2}y+xy^{2}\right),\lambda$为常数。求准确到二级微扰修正的基态能量.

(提示:粒子数表象下,$\left\langle n_{x}'\left|\hat{x}\right|n_{x}\right\rangle =\sqrt{\frac{\hbar}{2\mu\omega}}\left(\sqrt{n_{x}+1}\delta_{n_{x}',n_{x}+1}+\sqrt{n_{x}}\delta_{n_{x}',n_{x}-1}\right)$).


三.两个自旋为$\frac{1}{2}$的粒子组成的体系,$\boldsymbol{\hat{S_{1}}}$和$\boldsymbol{\hat{S_{2}}}$分别表示两个粒子的自旋算符,$\boldsymbol{\hat{S}}$为两粒子的总自旋算符,$\boldsymbol{\hat{\boldsymbol{n}}}$表示两粒子相对运动方向的单位矢量。设系统的相互作用哈密顿量为$H=3\left(\boldsymbol{\hat{S}_{1}}\cdot\boldsymbol{\hat{\boldsymbol{n}}}\right)\left(\boldsymbol{\hat{S}_{2}}\cdot\boldsymbol{\hat{\boldsymbol{n}}}\right)-\boldsymbol{\hat{S}}_{1}\cdot\boldsymbol{\hat{S}}_{2}$

1.证明

(a)$\boldsymbol{\hat{S}}_{1}\cdot\boldsymbol{\hat{S}}_{2}=\dfrac{1}{2}\boldsymbol{\hat{S^{2}}}-\dfrac{3}{4}\hbar^{2}$

(b)$\left(\boldsymbol{\hat{S}_{1}}\cdot\boldsymbol{\hat{\boldsymbol{n}}}\right)\left(\boldsymbol{\hat{S}_{2}}\cdot\boldsymbol{\hat{\boldsymbol{n}}}\right)=\dfrac{1}{2}\left(\boldsymbol{\hat{S}}\cdot\boldsymbol{\hat{n}}\right)^{2}-\dfrac{1}{4}\hbar^{2}$

2.证明$\left[H,\boldsymbol{\hat{S^{2}}}\right]=0$



四.质量为$\mu$的粒子在一维势$V\left(x\right)=\begin{cases}
\infty & x<0\\
Bx & x>0
\end{cases}$中运动,其中$B>0$是常数.

1.试从下列波函数中选择一个合理的束缚态试探波函数,并说明理由.

\begin{tabular}{cccccccccccccccccc}
a)$e^{-\frac{x}{a}}$ &  &  &  &  &  & b)$xe^{-\frac{x}{a}}$ &  &  &  &  &  & c)$1-e^{-\frac{x}{a}}$ &  &  &  &  & \\
\end{tabular}

其中$a>0$为变分参数.

2.取所选的试探波函数,用变分法估算体系基态能量.


五.一个二能级体系,哈密顿算符的矩阵表达式为$H_{0}=\left(\begin{array}{cc}
E_{1}^{\left(0\right)} & 0\\
0 &
E_{2}^{\left(0\right)}\end{array}\right)$,$\left(E_{1}^{\left(0\right)}<E_{2}^{\left(0\right)}\right)$.设$t=0$时刻体系处于$H_0$的基态上,后受微扰$H'$的作用,$H'=\left(\begin{array}{cc}
0 & \gamma\\
\gamma & 0
\end{array}\right)$,$\gamma$为常数.

1.求出在$t>0$时刻,体系处于$H_{0}$激发态的几率$P_{E_{2}^{\left(0\right)}}\left(t\right)$的精确表达式.

2.利用一阶含时微扰论求$P_{E_{2}^{\left(0\right)}}\left(t\right)$,并与精确表达式比较,讨论所得结果的适用条件.

\thispagestyle{empty}
\end{document}
